\FloatBarrier
\chapter{Conclusion}
\label{section:conclusion}

This paper surveyed different strategies for the simulation of deformable objects. A semi-elastic mass-spring model built from the triangular mesh of an object, which is used to represent deformable cloth and aid in collision handling/response, has been explained. A more recent and stable position-based approach has been studied. Different low-level overlap tests between triangles exist. Bounding Volume Hierarchies have been discussed, as they have been the subject of ample research. There are many possible implementations, which differ in the choice of bounding volume, construction, update/refitting, and what is stored at the leaf nodes. The concept of Representative Triangles can be used in combination with BVHs in order to further prune unnecessary primitive tests. RTs are built from a normal triangular mesh, and assign features (vertices, edges or faces) to incident triangles, and are then structured in a hierarchy. A different approach to collision detection using Signed Distance Fields has been presented. Most methods in this class are suited for off-line applications, but these distance fields can be used for more than intersection, such as ray casting and motion planning. A SDF implementation that works in interactive environments, for detecting collision of deformable objects against rigid objects, is discussed. A robust and efficient spatial hashing based collision detection method is explained, having various parameters that can be tweaked according to the simulation in order to achieve better performance. Adaptations of well known structures (normal cones) and properties to continuous collision detection are presented. An application has been developed, which implements a hybrid PBD - impulse cloth simulator on the CPU, and GPU parallelized Mass-Spring and pure PBD engines.

\chapter{Further Work}
\label{section:further}

Future research will focus on improving the performance of the BVH structure, along with implementing and optimizing the other CD methods surveyed. 

Another area of improvement in the application is the user interface. In its current stage, the OpenGL drawing window and the standard console are the only means of feedback, and interacting with the application is done by using a number of keybinds. In a future update, some of the functionality will be moved to a GUI.

Collision detection in real-time applications presents interest. The objective of future research is a robust intersection method that makes use of the advanced graphics hardware in present times, taking advantage of their massive parallel capabilities, focusing on highly interactive environments and consumer-level technology. GPU implementations both within and outside the graphics pipeline are to be developed. Continuous collision detection is preferred, as it is the "cleaner" and more modular approach: in the case of discrete collision detection, the response module must unrealistically transpose vertices in order to avoid inter-penetration and preserve realism. Further research into bounding volume hierarchies and the combination of these structures with other strategies will be done.