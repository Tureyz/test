\chapter{Introduction}
\label{section:intro}

Modeling a realistic scene through a computer graphics program has been a subject of research for many years. Applications include virtual surgery, film making, video games, automobile design etc. There are many components in a graphic engine which contribute to the realism of the rendered scene: lightning, shadows, particle effects among others. Another important module, and the focus of this paper, is the collision detection (CD). A simulation must find out any intersection between the objects present in the scene, which are often composed of thousands of primitives, such as triangles. This module can quickly become the bottleneck of the application, as the naive implementation gives $O(n^{2})$ complexity. In order to accelerate this process, CD is usually performed in two phases: the broad phase, which quickly eliminates pairs of objects that cannot intersect, and the narrow phase, which uses this output to perform accurate elementary tests on the candidate object pairs. The resulting primitive pairs, and often times additional information related to the collision, such as penetration depth and direction, are then fed into the collision response module, which determines the behavior for the following time interval, taking into account factors like friction, gravity, density, wind speed, and restitution.

Initially, most collision detection methods focused on rigid bodies. However, in recent times, a greater focus has been put on deformable objects. One of the main applications is cloth simulation, and besides checking for collisions between a deformable object and other elements present in the scene, the collision detection method also has to check for self-collisions: intersections between different regions of the same deformable object.

This paper discusses the different approaches to deformable body collision detection. The methods presented differ in a variety of ways:

\paragraph{Discrete vs. Continuous.}

Collision detection methods can be further categorized into discrete and continuous methods. The former only considers discrete time steps in which bodies are checked for overlap. This offers better performance, but intersections rarely occur at exactly the times in which the physics is updated, which causes objects to inter-penetrate. In contrast, continuous collision detection can find the exact time of contact between two time frames, and is evidently more demanding in terms of complexity.

\paragraph{Performance vs. Accuracy.}

Some methods are more suited for off-line applications, with no interactivity, offering lifelike results but at the cost of a greater time-per-frame. Other approaches sacrifice the visual fidelity in favor of performance, and are used in real-time applications.

\paragraph{Object Topology and Representation.}
Some methods require a specific type of topology, such as tetrahedral meshes. Another scenario that must be accurately simulated is object destruction, such as tearing of cloth. Not all CD methods deal with this case.

