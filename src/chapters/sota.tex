\chapter{State Of The Art}
\label{chapter:sota}

In applications modeling virtual scenes, objects (rigid or otherwise) are often represented as triangular meshes. In order to give realistic deformable physical properties to these triangular meshes, which are sometimes infinitely thin (in the case of cloth), these meshes are enhanced with a semi-elastic model described in \cite{provot95}. In this system using masses and springs, each vertex holds a virtual mass. The masses are linked by three types of springs. Describing the topology of the masses in terms of a two dimensional array allows the definition of these springs. The first type are the structural springs, which connect masses $\mathbf{[i, j] <-> [i + 1, j]}$ and $\mathbf{[i, j] <-> [i, j + 1]}$. The second type, connecting masses $\mathbf{[i, j] <-> [i + 1, j + 1]}$ and $\mathbf{[i + 1, j] <-> [i, j + 1]}$ are the shear springs. Finally, the flexion springs link masses $\mathbf{[i, j] <-> [i + 2, j]}$ and $\mathbf{[i, j] <-> [i, j + 2]}$. All the springs are weightless and have a natural length. The behavior of such a model is given my Newton's third law, $F = m * a$. The total force acting on every virtual mass can be split into two components: internal forces, such as the tension between the three types of springs, and external forces, such as gravity.

\todo{La deformabile conteaza si self-collision}


\section{Continuous Collision Detection}
\label{sub-sec:ccd}

Most rigid body simulations approach collision detection in a discrete manner. \textbf{Discrete collision detection} algorithms require a list of objects in the broad phase, or a list of primitives in the more precise mid and narrow phases. These lists are updated at each simulation step, and they are used to find those pairs of intersecting elements. This approach is more computationally efficient, but it has one major drawback, which makes it unfeasible for deformable objects. Due to the (most commonly fixed) time step at which the physics simulation is updated, the exact point of contact between two elements is unknown. The collision detection result should be interpreted as "elements \textbf{A} and \textbf{B} have intersected sometime between steps \textbf{T} and \textbf{T + 1}". This allows the "bullet through paper" problem, also known as \textbf{tunneling}, to occur: An object \textbf{A} is heading towards another, \textbf{B}, with such speed that at time step \textbf{T} they have not yet intersected, but at the next step, \textbf{T + 1}, \textbf{A} has already completely passed through \textbf{B}, and the collision is missed. One could shorten the time between simulation updates, but this filters only some missed collisions, not all. In addition, this approach negatively impacts application performance. Depending on the level of object detail, these inter-penetrations happen frequently, even with small time windows between simulation updates. For deformable objects, this is unacceptable, as the simulation might never recover ("sewn" cloth due to missed collisions that can never untangle), forfeiting life-like behavior. For this reason, deformable object simulations often implement \textbf{continuous collision detection} methods. Unlike their discrete counterparts, these methods can accurately detect the intersection between two elements. Between two consecutive time steps, the simulation interpolates the position of all elements, and the actual detection is done on the deformed volumes of these swept objects.

In \cite{provot97}, Provot uses the earlier defined mass-spring model \cite{provot95} to perform continuous collision detection. It is shown that collisions can only be of two types. The first type is the \textbf{point-triangle} collision - a mesh vertex is in contact with a triangle (either from the same mesh, or a different one). The second is the \textbf{edge-edge} collision, in which two different edges (which again, can be part of the same mesh) collide.

In order to detect point-triangle collisions, a vectorial equation giving a non-linear system must be solved. However, the following condition can be used: $\overrightarrow{AP}(t) \cdot \overrightarrow{N} = 0$, where $A$ is a vertex of the triangle, $P$ is the point and $N$ is a normal to the plane defined by the triangle. This dot product is a third degree equation. Of the three $t$ results, only those that lie in the current time interval can correspond to collisions. These values are then substituted in the initial vectorial equation. If more than one are solutions to the system, only the one belonging to the smallest value of t (the one that happened the earliest) is considered to be a collision \cite{provot97}.

\todo{Mai mult}

\section{Bounding Volume Hierarchies}
\label{sub-sec:bvh}

\todo{Construire BVH, update si traversare}
\todo{Ce se stocheaza practic in BVH - mai multe posibilitati}

\section{Signed Distance Fields}
\label{sub-sec:sdf}

\section{Uniform Spatial Partitioning}
\label{sub-sec:usp}

